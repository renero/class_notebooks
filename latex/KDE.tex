\documentclass[11pt]{article}

    \usepackage[breakable]{tcolorbox}
    \usepackage{parskip} % Stop auto-indenting (to mimic markdown behaviour)
    
    \usepackage{iftex}
    \ifPDFTeX
    	\usepackage[T1]{fontenc}
    	\usepackage{mathpazo}
    \else
    	\usepackage{fontspec}
    \fi

    % Basic figure setup, for now with no caption control since it's done
    % automatically by Pandoc (which extracts ![](path) syntax from Markdown).
    \usepackage{graphicx}
    % Maintain compatibility with old templates. Remove in nbconvert 6.0
    \let\Oldincludegraphics\includegraphics
    % Ensure that by default, figures have no caption (until we provide a
    % proper Figure object with a Caption API and a way to capture that
    % in the conversion process - todo).
    \usepackage{caption}
    \DeclareCaptionFormat{nocaption}{}
    \captionsetup{format=nocaption,aboveskip=0pt,belowskip=0pt}

    \usepackage[Export]{adjustbox} % Used to constrain images to a maximum size
    \adjustboxset{max size={0.9\linewidth}{0.9\paperheight}}
    \usepackage{float}
    \floatplacement{figure}{H} % forces figures to be placed at the correct location
    \usepackage{xcolor} % Allow colors to be defined
    \usepackage{enumerate} % Needed for markdown enumerations to work
    \usepackage{geometry} % Used to adjust the document margins
    \usepackage{amsmath} % Equations
    \usepackage{amssymb} % Equations
    \usepackage{textcomp} % defines textquotesingle
    % Hack from http://tex.stackexchange.com/a/47451/13684:
    \AtBeginDocument{%
        \def\PYZsq{\textquotesingle}% Upright quotes in Pygmentized code
    }
    \usepackage{upquote} % Upright quotes for verbatim code
    \usepackage{eurosym} % defines \euro
    \usepackage[mathletters]{ucs} % Extended unicode (utf-8) support
    \usepackage{fancyvrb} % verbatim replacement that allows latex
    \usepackage{grffile} % extends the file name processing of package graphics 
                         % to support a larger range
    \makeatletter % fix for grffile with XeLaTeX
    \def\Gread@@xetex#1{%
      \IfFileExists{"\Gin@base".bb}%
      {\Gread@eps{\Gin@base.bb}}%
      {\Gread@@xetex@aux#1}%
    }
    \makeatother

    % The hyperref package gives us a pdf with properly built
    % internal navigation ('pdf bookmarks' for the table of contents,
    % internal cross-reference links, web links for URLs, etc.)
    \usepackage{hyperref}
    % The default LaTeX title has an obnoxious amount of whitespace. By default,
    % titling removes some of it. It also provides customization options.
    \usepackage{titling}
    \usepackage{longtable} % longtable support required by pandoc >1.10
    \usepackage{booktabs}  % table support for pandoc > 1.12.2
    \usepackage[inline]{enumitem} % IRkernel/repr support (it uses the enumerate* environment)
    \usepackage[normalem]{ulem} % ulem is needed to support strikethroughs (\sout)
                                % normalem makes italics be italics, not underlines
    \usepackage{mathrsfs}
    

    
    % Colors for the hyperref package
    \definecolor{urlcolor}{rgb}{0,.145,.698}
    \definecolor{linkcolor}{rgb}{.71,0.21,0.01}
    \definecolor{citecolor}{rgb}{.12,.54,.11}

    % ANSI colors
    \definecolor{ansi-black}{HTML}{3E424D}
    \definecolor{ansi-black-intense}{HTML}{282C36}
    \definecolor{ansi-red}{HTML}{E75C58}
    \definecolor{ansi-red-intense}{HTML}{B22B31}
    \definecolor{ansi-green}{HTML}{00A250}
    \definecolor{ansi-green-intense}{HTML}{007427}
    \definecolor{ansi-yellow}{HTML}{DDB62B}
    \definecolor{ansi-yellow-intense}{HTML}{B27D12}
    \definecolor{ansi-blue}{HTML}{208FFB}
    \definecolor{ansi-blue-intense}{HTML}{0065CA}
    \definecolor{ansi-magenta}{HTML}{D160C4}
    \definecolor{ansi-magenta-intense}{HTML}{A03196}
    \definecolor{ansi-cyan}{HTML}{60C6C8}
    \definecolor{ansi-cyan-intense}{HTML}{258F8F}
    \definecolor{ansi-white}{HTML}{C5C1B4}
    \definecolor{ansi-white-intense}{HTML}{A1A6B2}
    \definecolor{ansi-default-inverse-fg}{HTML}{FFFFFF}
    \definecolor{ansi-default-inverse-bg}{HTML}{000000}

    % commands and environments needed by pandoc snippets
    % extracted from the output of `pandoc -s`
    \providecommand{\tightlist}{%
      \setlength{\itemsep}{0pt}\setlength{\parskip}{0pt}}
    \DefineVerbatimEnvironment{Highlighting}{Verbatim}{commandchars=\\\{\}}
    % Add ',fontsize=\small' for more characters per line
    \newenvironment{Shaded}{}{}
    \newcommand{\KeywordTok}[1]{\textcolor[rgb]{0.00,0.44,0.13}{\textbf{{#1}}}}
    \newcommand{\DataTypeTok}[1]{\textcolor[rgb]{0.56,0.13,0.00}{{#1}}}
    \newcommand{\DecValTok}[1]{\textcolor[rgb]{0.25,0.63,0.44}{{#1}}}
    \newcommand{\BaseNTok}[1]{\textcolor[rgb]{0.25,0.63,0.44}{{#1}}}
    \newcommand{\FloatTok}[1]{\textcolor[rgb]{0.25,0.63,0.44}{{#1}}}
    \newcommand{\CharTok}[1]{\textcolor[rgb]{0.25,0.44,0.63}{{#1}}}
    \newcommand{\StringTok}[1]{\textcolor[rgb]{0.25,0.44,0.63}{{#1}}}
    \newcommand{\CommentTok}[1]{\textcolor[rgb]{0.38,0.63,0.69}{\textit{{#1}}}}
    \newcommand{\OtherTok}[1]{\textcolor[rgb]{0.00,0.44,0.13}{{#1}}}
    \newcommand{\AlertTok}[1]{\textcolor[rgb]{1.00,0.00,0.00}{\textbf{{#1}}}}
    \newcommand{\FunctionTok}[1]{\textcolor[rgb]{0.02,0.16,0.49}{{#1}}}
    \newcommand{\RegionMarkerTok}[1]{{#1}}
    \newcommand{\ErrorTok}[1]{\textcolor[rgb]{1.00,0.00,0.00}{\textbf{{#1}}}}
    \newcommand{\NormalTok}[1]{{#1}}
    
    % Additional commands for more recent versions of Pandoc
    \newcommand{\ConstantTok}[1]{\textcolor[rgb]{0.53,0.00,0.00}{{#1}}}
    \newcommand{\SpecialCharTok}[1]{\textcolor[rgb]{0.25,0.44,0.63}{{#1}}}
    \newcommand{\VerbatimStringTok}[1]{\textcolor[rgb]{0.25,0.44,0.63}{{#1}}}
    \newcommand{\SpecialStringTok}[1]{\textcolor[rgb]{0.73,0.40,0.53}{{#1}}}
    \newcommand{\ImportTok}[1]{{#1}}
    \newcommand{\DocumentationTok}[1]{\textcolor[rgb]{0.73,0.13,0.13}{\textit{{#1}}}}
    \newcommand{\AnnotationTok}[1]{\textcolor[rgb]{0.38,0.63,0.69}{\textbf{\textit{{#1}}}}}
    \newcommand{\CommentVarTok}[1]{\textcolor[rgb]{0.38,0.63,0.69}{\textbf{\textit{{#1}}}}}
    \newcommand{\VariableTok}[1]{\textcolor[rgb]{0.10,0.09,0.49}{{#1}}}
    \newcommand{\ControlFlowTok}[1]{\textcolor[rgb]{0.00,0.44,0.13}{\textbf{{#1}}}}
    \newcommand{\OperatorTok}[1]{\textcolor[rgb]{0.40,0.40,0.40}{{#1}}}
    \newcommand{\BuiltInTok}[1]{{#1}}
    \newcommand{\ExtensionTok}[1]{{#1}}
    \newcommand{\PreprocessorTok}[1]{\textcolor[rgb]{0.74,0.48,0.00}{{#1}}}
    \newcommand{\AttributeTok}[1]{\textcolor[rgb]{0.49,0.56,0.16}{{#1}}}
    \newcommand{\InformationTok}[1]{\textcolor[rgb]{0.38,0.63,0.69}{\textbf{\textit{{#1}}}}}
    \newcommand{\WarningTok}[1]{\textcolor[rgb]{0.38,0.63,0.69}{\textbf{\textit{{#1}}}}}
    
    
    % Define a nice break command that doesn't care if a line doesn't already
    % exist.
    \def\br{\hspace*{\fill} \\* }
    % Math Jax compatibility definitions
    \def\gt{>}
    \def\lt{<}
    \let\Oldtex\TeX
    \let\Oldlatex\LaTeX
    \renewcommand{\TeX}{\textrm{\Oldtex}}
    \renewcommand{\LaTeX}{\textrm{\Oldlatex}}
    % Document parameters
    % Document title
    \title{KDE}
    
    
    
    
    
% Pygments definitions
\makeatletter
\def\PY@reset{\let\PY@it=\relax \let\PY@bf=\relax%
    \let\PY@ul=\relax \let\PY@tc=\relax%
    \let\PY@bc=\relax \let\PY@ff=\relax}
\def\PY@tok#1{\csname PY@tok@#1\endcsname}
\def\PY@toks#1+{\ifx\relax#1\empty\else%
    \PY@tok{#1}\expandafter\PY@toks\fi}
\def\PY@do#1{\PY@bc{\PY@tc{\PY@ul{%
    \PY@it{\PY@bf{\PY@ff{#1}}}}}}}
\def\PY#1#2{\PY@reset\PY@toks#1+\relax+\PY@do{#2}}

\expandafter\def\csname PY@tok@w\endcsname{\def\PY@tc##1{\textcolor[rgb]{0.73,0.73,0.73}{##1}}}
\expandafter\def\csname PY@tok@c\endcsname{\let\PY@it=\textit\def\PY@tc##1{\textcolor[rgb]{0.25,0.50,0.50}{##1}}}
\expandafter\def\csname PY@tok@cp\endcsname{\def\PY@tc##1{\textcolor[rgb]{0.74,0.48,0.00}{##1}}}
\expandafter\def\csname PY@tok@k\endcsname{\let\PY@bf=\textbf\def\PY@tc##1{\textcolor[rgb]{0.00,0.50,0.00}{##1}}}
\expandafter\def\csname PY@tok@kp\endcsname{\def\PY@tc##1{\textcolor[rgb]{0.00,0.50,0.00}{##1}}}
\expandafter\def\csname PY@tok@kt\endcsname{\def\PY@tc##1{\textcolor[rgb]{0.69,0.00,0.25}{##1}}}
\expandafter\def\csname PY@tok@o\endcsname{\def\PY@tc##1{\textcolor[rgb]{0.40,0.40,0.40}{##1}}}
\expandafter\def\csname PY@tok@ow\endcsname{\let\PY@bf=\textbf\def\PY@tc##1{\textcolor[rgb]{0.67,0.13,1.00}{##1}}}
\expandafter\def\csname PY@tok@nb\endcsname{\def\PY@tc##1{\textcolor[rgb]{0.00,0.50,0.00}{##1}}}
\expandafter\def\csname PY@tok@nf\endcsname{\def\PY@tc##1{\textcolor[rgb]{0.00,0.00,1.00}{##1}}}
\expandafter\def\csname PY@tok@nc\endcsname{\let\PY@bf=\textbf\def\PY@tc##1{\textcolor[rgb]{0.00,0.00,1.00}{##1}}}
\expandafter\def\csname PY@tok@nn\endcsname{\let\PY@bf=\textbf\def\PY@tc##1{\textcolor[rgb]{0.00,0.00,1.00}{##1}}}
\expandafter\def\csname PY@tok@ne\endcsname{\let\PY@bf=\textbf\def\PY@tc##1{\textcolor[rgb]{0.82,0.25,0.23}{##1}}}
\expandafter\def\csname PY@tok@nv\endcsname{\def\PY@tc##1{\textcolor[rgb]{0.10,0.09,0.49}{##1}}}
\expandafter\def\csname PY@tok@no\endcsname{\def\PY@tc##1{\textcolor[rgb]{0.53,0.00,0.00}{##1}}}
\expandafter\def\csname PY@tok@nl\endcsname{\def\PY@tc##1{\textcolor[rgb]{0.63,0.63,0.00}{##1}}}
\expandafter\def\csname PY@tok@ni\endcsname{\let\PY@bf=\textbf\def\PY@tc##1{\textcolor[rgb]{0.60,0.60,0.60}{##1}}}
\expandafter\def\csname PY@tok@na\endcsname{\def\PY@tc##1{\textcolor[rgb]{0.49,0.56,0.16}{##1}}}
\expandafter\def\csname PY@tok@nt\endcsname{\let\PY@bf=\textbf\def\PY@tc##1{\textcolor[rgb]{0.00,0.50,0.00}{##1}}}
\expandafter\def\csname PY@tok@nd\endcsname{\def\PY@tc##1{\textcolor[rgb]{0.67,0.13,1.00}{##1}}}
\expandafter\def\csname PY@tok@s\endcsname{\def\PY@tc##1{\textcolor[rgb]{0.73,0.13,0.13}{##1}}}
\expandafter\def\csname PY@tok@sd\endcsname{\let\PY@it=\textit\def\PY@tc##1{\textcolor[rgb]{0.73,0.13,0.13}{##1}}}
\expandafter\def\csname PY@tok@si\endcsname{\let\PY@bf=\textbf\def\PY@tc##1{\textcolor[rgb]{0.73,0.40,0.53}{##1}}}
\expandafter\def\csname PY@tok@se\endcsname{\let\PY@bf=\textbf\def\PY@tc##1{\textcolor[rgb]{0.73,0.40,0.13}{##1}}}
\expandafter\def\csname PY@tok@sr\endcsname{\def\PY@tc##1{\textcolor[rgb]{0.73,0.40,0.53}{##1}}}
\expandafter\def\csname PY@tok@ss\endcsname{\def\PY@tc##1{\textcolor[rgb]{0.10,0.09,0.49}{##1}}}
\expandafter\def\csname PY@tok@sx\endcsname{\def\PY@tc##1{\textcolor[rgb]{0.00,0.50,0.00}{##1}}}
\expandafter\def\csname PY@tok@m\endcsname{\def\PY@tc##1{\textcolor[rgb]{0.40,0.40,0.40}{##1}}}
\expandafter\def\csname PY@tok@gh\endcsname{\let\PY@bf=\textbf\def\PY@tc##1{\textcolor[rgb]{0.00,0.00,0.50}{##1}}}
\expandafter\def\csname PY@tok@gu\endcsname{\let\PY@bf=\textbf\def\PY@tc##1{\textcolor[rgb]{0.50,0.00,0.50}{##1}}}
\expandafter\def\csname PY@tok@gd\endcsname{\def\PY@tc##1{\textcolor[rgb]{0.63,0.00,0.00}{##1}}}
\expandafter\def\csname PY@tok@gi\endcsname{\def\PY@tc##1{\textcolor[rgb]{0.00,0.63,0.00}{##1}}}
\expandafter\def\csname PY@tok@gr\endcsname{\def\PY@tc##1{\textcolor[rgb]{1.00,0.00,0.00}{##1}}}
\expandafter\def\csname PY@tok@ge\endcsname{\let\PY@it=\textit}
\expandafter\def\csname PY@tok@gs\endcsname{\let\PY@bf=\textbf}
\expandafter\def\csname PY@tok@gp\endcsname{\let\PY@bf=\textbf\def\PY@tc##1{\textcolor[rgb]{0.00,0.00,0.50}{##1}}}
\expandafter\def\csname PY@tok@go\endcsname{\def\PY@tc##1{\textcolor[rgb]{0.53,0.53,0.53}{##1}}}
\expandafter\def\csname PY@tok@gt\endcsname{\def\PY@tc##1{\textcolor[rgb]{0.00,0.27,0.87}{##1}}}
\expandafter\def\csname PY@tok@err\endcsname{\def\PY@bc##1{\setlength{\fboxsep}{0pt}\fcolorbox[rgb]{1.00,0.00,0.00}{1,1,1}{\strut ##1}}}
\expandafter\def\csname PY@tok@kc\endcsname{\let\PY@bf=\textbf\def\PY@tc##1{\textcolor[rgb]{0.00,0.50,0.00}{##1}}}
\expandafter\def\csname PY@tok@kd\endcsname{\let\PY@bf=\textbf\def\PY@tc##1{\textcolor[rgb]{0.00,0.50,0.00}{##1}}}
\expandafter\def\csname PY@tok@kn\endcsname{\let\PY@bf=\textbf\def\PY@tc##1{\textcolor[rgb]{0.00,0.50,0.00}{##1}}}
\expandafter\def\csname PY@tok@kr\endcsname{\let\PY@bf=\textbf\def\PY@tc##1{\textcolor[rgb]{0.00,0.50,0.00}{##1}}}
\expandafter\def\csname PY@tok@bp\endcsname{\def\PY@tc##1{\textcolor[rgb]{0.00,0.50,0.00}{##1}}}
\expandafter\def\csname PY@tok@fm\endcsname{\def\PY@tc##1{\textcolor[rgb]{0.00,0.00,1.00}{##1}}}
\expandafter\def\csname PY@tok@vc\endcsname{\def\PY@tc##1{\textcolor[rgb]{0.10,0.09,0.49}{##1}}}
\expandafter\def\csname PY@tok@vg\endcsname{\def\PY@tc##1{\textcolor[rgb]{0.10,0.09,0.49}{##1}}}
\expandafter\def\csname PY@tok@vi\endcsname{\def\PY@tc##1{\textcolor[rgb]{0.10,0.09,0.49}{##1}}}
\expandafter\def\csname PY@tok@vm\endcsname{\def\PY@tc##1{\textcolor[rgb]{0.10,0.09,0.49}{##1}}}
\expandafter\def\csname PY@tok@sa\endcsname{\def\PY@tc##1{\textcolor[rgb]{0.73,0.13,0.13}{##1}}}
\expandafter\def\csname PY@tok@sb\endcsname{\def\PY@tc##1{\textcolor[rgb]{0.73,0.13,0.13}{##1}}}
\expandafter\def\csname PY@tok@sc\endcsname{\def\PY@tc##1{\textcolor[rgb]{0.73,0.13,0.13}{##1}}}
\expandafter\def\csname PY@tok@dl\endcsname{\def\PY@tc##1{\textcolor[rgb]{0.73,0.13,0.13}{##1}}}
\expandafter\def\csname PY@tok@s2\endcsname{\def\PY@tc##1{\textcolor[rgb]{0.73,0.13,0.13}{##1}}}
\expandafter\def\csname PY@tok@sh\endcsname{\def\PY@tc##1{\textcolor[rgb]{0.73,0.13,0.13}{##1}}}
\expandafter\def\csname PY@tok@s1\endcsname{\def\PY@tc##1{\textcolor[rgb]{0.73,0.13,0.13}{##1}}}
\expandafter\def\csname PY@tok@mb\endcsname{\def\PY@tc##1{\textcolor[rgb]{0.40,0.40,0.40}{##1}}}
\expandafter\def\csname PY@tok@mf\endcsname{\def\PY@tc##1{\textcolor[rgb]{0.40,0.40,0.40}{##1}}}
\expandafter\def\csname PY@tok@mh\endcsname{\def\PY@tc##1{\textcolor[rgb]{0.40,0.40,0.40}{##1}}}
\expandafter\def\csname PY@tok@mi\endcsname{\def\PY@tc##1{\textcolor[rgb]{0.40,0.40,0.40}{##1}}}
\expandafter\def\csname PY@tok@il\endcsname{\def\PY@tc##1{\textcolor[rgb]{0.40,0.40,0.40}{##1}}}
\expandafter\def\csname PY@tok@mo\endcsname{\def\PY@tc##1{\textcolor[rgb]{0.40,0.40,0.40}{##1}}}
\expandafter\def\csname PY@tok@ch\endcsname{\let\PY@it=\textit\def\PY@tc##1{\textcolor[rgb]{0.25,0.50,0.50}{##1}}}
\expandafter\def\csname PY@tok@cm\endcsname{\let\PY@it=\textit\def\PY@tc##1{\textcolor[rgb]{0.25,0.50,0.50}{##1}}}
\expandafter\def\csname PY@tok@cpf\endcsname{\let\PY@it=\textit\def\PY@tc##1{\textcolor[rgb]{0.25,0.50,0.50}{##1}}}
\expandafter\def\csname PY@tok@c1\endcsname{\let\PY@it=\textit\def\PY@tc##1{\textcolor[rgb]{0.25,0.50,0.50}{##1}}}
\expandafter\def\csname PY@tok@cs\endcsname{\let\PY@it=\textit\def\PY@tc##1{\textcolor[rgb]{0.25,0.50,0.50}{##1}}}

\def\PYZbs{\char`\\}
\def\PYZus{\char`\_}
\def\PYZob{\char`\{}
\def\PYZcb{\char`\}}
\def\PYZca{\char`\^}
\def\PYZam{\char`\&}
\def\PYZlt{\char`\<}
\def\PYZgt{\char`\>}
\def\PYZsh{\char`\#}
\def\PYZpc{\char`\%}
\def\PYZdl{\char`\$}
\def\PYZhy{\char`\-}
\def\PYZsq{\char`\'}
\def\PYZdq{\char`\"}
\def\PYZti{\char`\~}
% for compatibility with earlier versions
\def\PYZat{@}
\def\PYZlb{[}
\def\PYZrb{]}
\makeatother


    % For linebreaks inside Verbatim environment from package fancyvrb. 
    \makeatletter
        \newbox\Wrappedcontinuationbox 
        \newbox\Wrappedvisiblespacebox 
        \newcommand*\Wrappedvisiblespace {\textcolor{red}{\textvisiblespace}} 
        \newcommand*\Wrappedcontinuationsymbol {\textcolor{red}{\llap{\tiny$\m@th\hookrightarrow$}}} 
        \newcommand*\Wrappedcontinuationindent {3ex } 
        \newcommand*\Wrappedafterbreak {\kern\Wrappedcontinuationindent\copy\Wrappedcontinuationbox} 
        % Take advantage of the already applied Pygments mark-up to insert 
        % potential linebreaks for TeX processing. 
        %        {, <, #, %, $, ' and ": go to next line. 
        %        _, }, ^, &, >, - and ~: stay at end of broken line. 
        % Use of \textquotesingle for straight quote. 
        \newcommand*\Wrappedbreaksatspecials {% 
            \def\PYGZus{\discretionary{\char`\_}{\Wrappedafterbreak}{\char`\_}}% 
            \def\PYGZob{\discretionary{}{\Wrappedafterbreak\char`\{}{\char`\{}}% 
            \def\PYGZcb{\discretionary{\char`\}}{\Wrappedafterbreak}{\char`\}}}% 
            \def\PYGZca{\discretionary{\char`\^}{\Wrappedafterbreak}{\char`\^}}% 
            \def\PYGZam{\discretionary{\char`\&}{\Wrappedafterbreak}{\char`\&}}% 
            \def\PYGZlt{\discretionary{}{\Wrappedafterbreak\char`\<}{\char`\<}}% 
            \def\PYGZgt{\discretionary{\char`\>}{\Wrappedafterbreak}{\char`\>}}% 
            \def\PYGZsh{\discretionary{}{\Wrappedafterbreak\char`\#}{\char`\#}}% 
            \def\PYGZpc{\discretionary{}{\Wrappedafterbreak\char`\%}{\char`\%}}% 
            \def\PYGZdl{\discretionary{}{\Wrappedafterbreak\char`\$}{\char`\$}}% 
            \def\PYGZhy{\discretionary{\char`\-}{\Wrappedafterbreak}{\char`\-}}% 
            \def\PYGZsq{\discretionary{}{\Wrappedafterbreak\textquotesingle}{\textquotesingle}}% 
            \def\PYGZdq{\discretionary{}{\Wrappedafterbreak\char`\"}{\char`\"}}% 
            \def\PYGZti{\discretionary{\char`\~}{\Wrappedafterbreak}{\char`\~}}% 
        } 
        % Some characters . , ; ? ! / are not pygmentized. 
        % This macro makes them "active" and they will insert potential linebreaks 
        \newcommand*\Wrappedbreaksatpunct {% 
            \lccode`\~`\.\lowercase{\def~}{\discretionary{\hbox{\char`\.}}{\Wrappedafterbreak}{\hbox{\char`\.}}}% 
            \lccode`\~`\,\lowercase{\def~}{\discretionary{\hbox{\char`\,}}{\Wrappedafterbreak}{\hbox{\char`\,}}}% 
            \lccode`\~`\;\lowercase{\def~}{\discretionary{\hbox{\char`\;}}{\Wrappedafterbreak}{\hbox{\char`\;}}}% 
            \lccode`\~`\:\lowercase{\def~}{\discretionary{\hbox{\char`\:}}{\Wrappedafterbreak}{\hbox{\char`\:}}}% 
            \lccode`\~`\?\lowercase{\def~}{\discretionary{\hbox{\char`\?}}{\Wrappedafterbreak}{\hbox{\char`\?}}}% 
            \lccode`\~`\!\lowercase{\def~}{\discretionary{\hbox{\char`\!}}{\Wrappedafterbreak}{\hbox{\char`\!}}}% 
            \lccode`\~`\/\lowercase{\def~}{\discretionary{\hbox{\char`\/}}{\Wrappedafterbreak}{\hbox{\char`\/}}}% 
            \catcode`\.\active
            \catcode`\,\active 
            \catcode`\;\active
            \catcode`\:\active
            \catcode`\?\active
            \catcode`\!\active
            \catcode`\/\active 
            \lccode`\~`\~ 	
        }
    \makeatother

    \let\OriginalVerbatim=\Verbatim
    \makeatletter
    \renewcommand{\Verbatim}[1][1]{%
        %\parskip\z@skip
        \sbox\Wrappedcontinuationbox {\Wrappedcontinuationsymbol}%
        \sbox\Wrappedvisiblespacebox {\FV@SetupFont\Wrappedvisiblespace}%
        \def\FancyVerbFormatLine ##1{\hsize\linewidth
            \vtop{\raggedright\hyphenpenalty\z@\exhyphenpenalty\z@
                \doublehyphendemerits\z@\finalhyphendemerits\z@
                \strut ##1\strut}%
        }%
        % If the linebreak is at a space, the latter will be displayed as visible
        % space at end of first line, and a continuation symbol starts next line.
        % Stretch/shrink are however usually zero for typewriter font.
        \def\FV@Space {%
            \nobreak\hskip\z@ plus\fontdimen3\font minus\fontdimen4\font
            \discretionary{\copy\Wrappedvisiblespacebox}{\Wrappedafterbreak}
            {\kern\fontdimen2\font}%
        }%
        
        % Allow breaks at special characters using \PYG... macros.
        \Wrappedbreaksatspecials
        % Breaks at punctuation characters . , ; ? ! and / need catcode=\active 	
        \OriginalVerbatim[#1,codes*=\Wrappedbreaksatpunct]%
    }
    \makeatother

    % Exact colors from NB
    \definecolor{incolor}{HTML}{303F9F}
    \definecolor{outcolor}{HTML}{D84315}
    \definecolor{cellborder}{HTML}{CFCFCF}
    \definecolor{cellbackground}{HTML}{F7F7F7}
    
    % prompt
    \makeatletter
    \newcommand{\boxspacing}{\kern\kvtcb@left@rule\kern\kvtcb@boxsep}
    \makeatother
    \newcommand{\prompt}[4]{
        \ttfamily\llap{{\color{#2}[#3]:\hspace{3pt}#4}}\vspace{-\baselineskip}
    }
    

    
    % Prevent overflowing lines due to hard-to-break entities
    \sloppy 
    % Setup hyperref package
    \hypersetup{
      breaklinks=true,  % so long urls are correctly broken across lines
      colorlinks=true,
      urlcolor=urlcolor,
      linkcolor=linkcolor,
      citecolor=citecolor,
      }
    % Slightly bigger margins than the latex defaults
    
    \geometry{verbose,tmargin=1in,bmargin=1in,lmargin=1in,rmargin=1in}
    
    

\begin{document}
    
    \maketitle
    
    

    
    \hypertarget{kde}{%
\section{KDE}\label{kde}}

Kernel Density Estimation (KDE) is a non-parametric way of estimating
the probability density function (pdf) of ANY distribution given a
finite number of its samples. The pdf of a random variable X given
finite samples (\(x_i\)), as per KDE formula, is given by:

\[ \text{pdf}(x)=\frac{1}{nh}\sum_{i=1}^{n} K ( \frac{||x - x_i||}{h} )\]

In the above equation, \(K()\) is called the kernel - a symmetric
function that integrates to \(1\).

\(h\) is called the \textbf{kernel bandwidth} which controls how smooth
(or spread-out) the kernel is.

The most commonly used kernel is the normal distribution function.

\[ K(x)=\frac{1}{\sigma \sqrt{2 \pi}} e^{-\frac{1}{2}\left(\frac{x-\mu}{\sigma}\right)^{2}} \]

The free parameters of kernel density estimation are the kernel, which
specifies the shape of the distribution placed at each point, and the
kernel bandwidth, which controls the size of the kernel at each point.
In practice, there are many kernels you might use for a kernel density
estimation: in particular, the Scikit-Learn KDE implementation
supports one of six kernels, which you can read about in Scikit-Learn's
Density Estimation documentation.

\href{https://jakevdp.github.io/PythonDataScienceHandbook/05.13-kernel-density-estimation.html}{source
1}

\href{http://shogun-toolbox.org/notebook/latest/KernelDensity.html}{source
2}

    \hypertarget{parameters}{%
\subsection{Parameters}\label{parameters}}

Setup all the parameters that I will need in later cells in a single
place.

    \begin{tcolorbox}[breakable, size=fbox, boxrule=1pt, pad at break*=1mm,colback=cellbackground, colframe=cellborder]
\prompt{In}{incolor}{2}{\boxspacing}
\begin{Verbatim}[commandchars=\\\{\}]
\PY{c+c1}{\PYZsh{} pdf function for plotting}
\PY{n}{n\PYZus{}samples} \PY{o}{=} \PY{l+m+mi}{100}
\PY{n}{x\PYZus{}min} \PY{o}{=} \PY{l+m+mf}{0.001}
\PY{n}{x\PYZus{}max} \PY{o}{=} \PY{l+m+mi}{15}

\PY{c+c1}{\PYZsh{} parameters of the distribution}
\PY{n}{mu1} \PY{o}{=} \PY{l+m+mi}{4}
\PY{n}{sigma1} \PY{o}{=} \PY{l+m+mi}{1}
\PY{n}{mu2} \PY{o}{=} \PY{l+m+mi}{8}
\PY{n}{sigma2} \PY{o}{=} \PY{l+m+mi}{2}

\PY{c+c1}{\PYZsh{} Parameters of the KDE}
\PY{n}{bandwidth} \PY{o}{=} \PY{l+m+mf}{0.5}
\end{Verbatim}
\end{tcolorbox}

    \hypertarget{random-data}{%
\subsection{Random data}\label{random-data}}

Let's generate a random set of points following a known distribution. We
will later try to approximate that ``unknown'' distribution using KDE.

    \begin{tcolorbox}[breakable, size=fbox, boxrule=1pt, pad at break*=1mm,colback=cellbackground, colframe=cellborder]
\prompt{In}{incolor}{3}{\boxspacing}
\begin{Verbatim}[commandchars=\\\{\}]
\PY{n}{x} \PY{o}{=} \PY{n}{np}\PY{o}{.}\PY{n}{linspace}\PY{p}{(}\PY{n}{x\PYZus{}min}\PY{p}{,} \PY{n}{x\PYZus{}max}\PY{p}{,} \PY{n}{n\PYZus{}samples}\PY{p}{)}
\PY{n}{y} \PY{o}{=} \PY{l+m+mf}{0.5}\PY{o}{*}\PY{p}{(}\PY{n}{stats}\PY{o}{.}\PY{n}{norm}\PY{p}{(}\PY{n}{mu1}\PY{p}{,} \PY{n}{sigma1}\PY{p}{)}\PY{o}{.}\PY{n}{pdf}\PY{p}{(}\PY{n}{x}\PY{p}{)}\PY{o}{+}\PY{n}{stats}\PY{o}{.}\PY{n}{norm}\PY{p}{(}\PY{n}{mu2}\PY{p}{,} \PY{n}{sigma2}\PY{p}{)}\PY{o}{.}\PY{n}{pdf}\PY{p}{(}\PY{n}{x}\PY{p}{)}\PY{p}{)}
\end{Verbatim}
\end{tcolorbox}

    \begin{center}
    \adjustimage{max size={0.9\linewidth}{0.9\paperheight}}{KDE_files/KDE_5_0.png}
    \end{center}
    { \hspace*{\fill} \\}
    
    Generate some samples with the parameters of the distributions proposed
above, and plot them over the X axis. Those are our dataset, and we have
to run KDE to estimate what is their distribution later.

    \begin{tcolorbox}[breakable, size=fbox, boxrule=1pt, pad at break*=1mm,colback=cellbackground, colframe=cellborder]
\prompt{In}{incolor}{4}{\boxspacing}
\begin{Verbatim}[commandchars=\\\{\}]
\PY{k}{def} \PY{n+nf}{generate\PYZus{}samples}\PY{p}{(}\PY{n}{n\PYZus{}samples}\PY{p}{,} \PY{n}{mu1}\PY{p}{,} \PY{n}{sigma1}\PY{p}{,} \PY{n}{mu2}\PY{p}{,} \PY{n}{sigma2}\PY{p}{)}\PY{p}{:}
    \PY{n}{samples1} \PY{o}{=} \PY{n}{np}\PY{o}{.}\PY{n}{random}\PY{o}{.}\PY{n}{normal}\PY{p}{(}\PY{n}{mu1}\PY{p}{,} \PY{n}{sigma1}\PY{p}{,} \PY{p}{(}\PY{l+m+mi}{1}\PY{p}{,} \PY{n+nb}{int}\PY{p}{(}\PY{n}{n\PYZus{}samples}\PY{o}{/}\PY{l+m+mi}{2}\PY{p}{)}\PY{p}{)}\PY{p}{)}
    \PY{n}{samples2} \PY{o}{=} \PY{n}{np}\PY{o}{.}\PY{n}{random}\PY{o}{.}\PY{n}{normal}\PY{p}{(}\PY{n}{mu2}\PY{p}{,} \PY{n}{sigma2}\PY{p}{,} \PY{p}{(}\PY{l+m+mi}{1}\PY{p}{,} \PY{n+nb}{int}\PY{p}{(}\PY{n}{n\PYZus{}samples}\PY{o}{/}\PY{l+m+mi}{2}\PY{p}{)}\PY{p}{)}\PY{p}{)}
    \PY{n}{samples} \PY{o}{=} \PY{n}{np}\PY{o}{.}\PY{n}{concatenate}\PY{p}{(}\PY{p}{(}\PY{n}{samples1}\PY{p}{,} \PY{n}{samples2}\PY{p}{)}\PY{p}{,} \PY{l+m+mi}{1}\PY{p}{)}
    \PY{k}{return} \PY{n}{samples}

\PY{n}{samples} \PY{o}{=} \PY{n}{generate\PYZus{}samples}\PY{p}{(}\PY{n}{n\PYZus{}samples}\PY{p}{,} \PY{n}{mu1}\PY{p}{,} \PY{n}{sigma1}\PY{p}{,} \PY{n}{mu2}\PY{p}{,} \PY{n}{sigma2}\PY{p}{)}
\PY{n}{samples} \PY{o}{=} \PY{n}{samples}\PY{o}{.}\PY{n}{T}
\end{Verbatim}
\end{tcolorbox}

    \begin{center}
    \adjustimage{max size={0.9\linewidth}{0.9\paperheight}}{KDE_files/KDE_7_0.png}
    \end{center}
    { \hspace*{\fill} \\}
    
    Run now our KDE estimate, using only 20 samples to see what it is
capable of doing.

    \begin{tcolorbox}[breakable, size=fbox, boxrule=1pt, pad at break*=1mm,colback=cellbackground, colframe=cellborder]
\prompt{In}{incolor}{5}{\boxspacing}
\begin{Verbatim}[commandchars=\\\{\}]
\PY{c+c1}{\PYZsh{} Generate more samples}
\PY{n}{n\PYZus{}samples} \PY{o}{=} \PY{l+m+mi}{20}
\PY{n}{samples} \PY{o}{=} \PY{n}{generate\PYZus{}samples}\PY{p}{(}\PY{n}{n\PYZus{}samples}\PY{p}{,} \PY{n}{mu1}\PY{p}{,} \PY{n}{sigma1}\PY{p}{,} \PY{n}{mu2}\PY{p}{,} \PY{n}{sigma2}\PY{p}{)}
\PY{n}{samples} \PY{o}{=} \PY{n}{samples}\PY{o}{.}\PY{n}{T}

\PY{c+c1}{\PYZsh{} instantiate and fit the KDE model}
\PY{n}{kde} \PY{o}{=} \PY{n}{KernelDensity}\PY{p}{(}\PY{n}{bandwidth}\PY{o}{=}\PY{n}{bandwidth}\PY{p}{,} \PY{n}{kernel}\PY{o}{=}\PY{l+s+s1}{\PYZsq{}}\PY{l+s+s1}{gaussian}\PY{l+s+s1}{\PYZsq{}}\PY{p}{)}
\PY{n}{kde}\PY{o}{.}\PY{n}{fit}\PY{p}{(}\PY{n}{samples}\PY{p}{)}

\PY{c+c1}{\PYZsh{} score\PYZus{}samples returns the log of the probability density}
\PY{c+c1}{\PYZsh{} `x\PYZus{}d` is a list of 20 equally spaced values between min and max}
\PY{n}{x\PYZus{}d} \PY{o}{=} \PY{n}{np}\PY{o}{.}\PY{n}{linspace}\PY{p}{(}\PY{n}{x\PYZus{}min}\PY{p}{,} \PY{n}{x\PYZus{}max}\PY{p}{,} \PY{n}{n\PYZus{}samples}\PY{p}{)}\PY{o}{.}\PY{n}{reshape}\PY{p}{(}\PY{o}{\PYZhy{}}\PY{l+m+mi}{1}\PY{p}{,} \PY{l+m+mi}{1}\PY{p}{)}
\PY{n}{logprob} \PY{o}{=} \PY{n}{kde}\PY{o}{.}\PY{n}{score\PYZus{}samples}\PY{p}{(}\PY{n}{x\PYZus{}d}\PY{p}{)}
\end{Verbatim}
\end{tcolorbox}

    \begin{center}
    \adjustimage{max size={0.9\linewidth}{0.9\paperheight}}{KDE_files/KDE_10_0.png}
    \end{center}
    { \hspace*{\fill} \\}
    
    As you can see, with such a low number of samples and the bandwidth
proposed, KDE makes a fair job in estimating what is the underlying
distribution of the data. If you want the red plot to better fit the
blue curve, increase the number of samples or play with the bandwidth,
and results will improve.

Remember that each value of the ``red'' KDE estimate is the
``estimated'' probability of each value of the X axis. You can see that
there're some regions where it is more likely to find values than in
some others. That difference in probabilities gives you the ability
(below) to build a classifier with this technique.

    \hypertarget{gridsearch-optimal-bandwith}{%
\subsection{GridSearch optimal
bandwith}\label{gridsearch-optimal-bandwith}}

Now, let's find what is the optimal value for the bandwidth to be used
in this specific example.

    \begin{tcolorbox}[breakable, size=fbox, boxrule=1pt, pad at break*=1mm,colback=cellbackground, colframe=cellborder]
\prompt{In}{incolor}{7}{\boxspacing}
\begin{Verbatim}[commandchars=\\\{\}]
\PY{n}{bandwidths} \PY{o}{=} \PY{l+m+mi}{10} \PY{o}{*}\PY{o}{*} \PY{n}{np}\PY{o}{.}\PY{n}{linspace}\PY{p}{(}\PY{o}{\PYZhy{}}\PY{l+m+mi}{1}\PY{p}{,} \PY{l+m+mi}{1}\PY{p}{,} \PY{l+m+mi}{100}\PY{p}{)}
\PY{n}{grid} \PY{o}{=} \PY{n}{GridSearchCV}\PY{p}{(}\PY{n}{KernelDensity}\PY{p}{(}\PY{n}{kernel}\PY{o}{=}\PY{l+s+s1}{\PYZsq{}}\PY{l+s+s1}{gaussian}\PY{l+s+s1}{\PYZsq{}}\PY{p}{)}\PY{p}{,}
                    \PY{p}{\PYZob{}}\PY{l+s+s1}{\PYZsq{}}\PY{l+s+s1}{bandwidth}\PY{l+s+s1}{\PYZsq{}}\PY{p}{:} \PY{n}{bandwidths}\PY{p}{\PYZcb{}}\PY{p}{,}
                    \PY{n}{cv}\PY{o}{=}\PY{n}{LeaveOneOut}\PY{p}{(}\PY{p}{)}\PY{p}{)}
\PY{n}{grid}\PY{o}{.}\PY{n}{fit}\PY{p}{(}\PY{n}{samples}\PY{p}{)}\PY{p}{;}
\PY{n}{bandwidth} \PY{o}{=} \PY{n}{grid}\PY{o}{.}\PY{n}{best\PYZus{}params\PYZus{}}\PY{p}{[}\PY{l+s+s1}{\PYZsq{}}\PY{l+s+s1}{bandwidth}\PY{l+s+s1}{\PYZsq{}}\PY{p}{]}
\PY{n+nb}{print}\PY{p}{(}\PY{l+s+s1}{\PYZsq{}}\PY{l+s+s1}{Optimal Bandwidth: }\PY{l+s+si}{\PYZob{}:.2f\PYZcb{}}\PY{l+s+s1}{\PYZsq{}}\PY{o}{.}\PY{n}{format}\PY{p}{(}\PY{n}{bandwidth}\PY{p}{)}\PY{p}{)}
\end{Verbatim}
\end{tcolorbox}

    \begin{Verbatim}[commandchars=\\\{\}]
Optimal Bandwidth: 0.81
    \end{Verbatim}

    \begin{tcolorbox}[breakable, size=fbox, boxrule=1pt, pad at break*=1mm,colback=cellbackground, colframe=cellborder]
\prompt{In}{incolor}{8}{\boxspacing}
\begin{Verbatim}[commandchars=\\\{\}]
\PY{c+c1}{\PYZsh{} instantiate and fit the KDE model over the samples}
\PY{n}{kde} \PY{o}{=} \PY{n}{KernelDensity}\PY{p}{(}\PY{n}{bandwidth}\PY{o}{=}\PY{n}{bandwidth}\PY{p}{,} \PY{n}{kernel}\PY{o}{=}\PY{l+s+s1}{\PYZsq{}}\PY{l+s+s1}{gaussian}\PY{l+s+s1}{\PYZsq{}}\PY{p}{)}
\PY{n}{kde}\PY{o}{.}\PY{n}{fit}\PY{p}{(}\PY{n}{samples}\PY{p}{)}

\PY{c+c1}{\PYZsh{} score\PYZus{}samples returns the log of the probability density}
\PY{n}{x\PYZus{}d} \PY{o}{=} \PY{n}{np}\PY{o}{.}\PY{n}{linspace}\PY{p}{(}\PY{n}{x\PYZus{}min}\PY{p}{,} \PY{n}{x\PYZus{}max}\PY{p}{,} \PY{n}{n\PYZus{}samples}\PY{p}{)}\PY{o}{.}\PY{n}{reshape}\PY{p}{(}\PY{o}{\PYZhy{}}\PY{l+m+mi}{1}\PY{p}{,} \PY{l+m+mi}{1}\PY{p}{)}
\PY{n}{logprob} \PY{o}{=} \PY{n}{kde}\PY{o}{.}\PY{n}{score\PYZus{}samples}\PY{p}{(}\PY{n}{x\PYZus{}d}\PY{p}{)}
\end{Verbatim}
\end{tcolorbox}

    \begin{center}
    \adjustimage{max size={0.9\linewidth}{0.9\paperheight}}{KDE_files/KDE_15_0.png}
    \end{center}
    { \hspace*{\fill} \\}
    
    As expected, with an optimal value for the bandwidth we get a better fit
of our KDE.

Check now what is the effect on the number of samples used to fit the
KDE. Let's jump from 20 to 200, with same bandwidth.

    \begin{tcolorbox}[breakable, size=fbox, boxrule=1pt, pad at break*=1mm,colback=cellbackground, colframe=cellborder]
\prompt{In}{incolor}{10}{\boxspacing}
\begin{Verbatim}[commandchars=\\\{\}]
\PY{c+c1}{\PYZsh{} Generate more samples}
\PY{n}{n\PYZus{}samples} \PY{o}{=} \PY{l+m+mi}{200}
\PY{n}{samples} \PY{o}{=} \PY{n}{generate\PYZus{}samples}\PY{p}{(}\PY{n}{n\PYZus{}samples}\PY{p}{,} \PY{n}{mu1}\PY{p}{,} \PY{n}{sigma1}\PY{p}{,} \PY{n}{mu2}\PY{p}{,} \PY{n}{sigma2}\PY{p}{)}
\PY{n}{samples} \PY{o}{=} \PY{n}{samples}\PY{o}{.}\PY{n}{T}

\PY{c+c1}{\PYZsh{} instantiate and fit the KDE model}
\PY{n}{kde} \PY{o}{=} \PY{n}{KernelDensity}\PY{p}{(}\PY{n}{bandwidth}\PY{o}{=}\PY{n}{bandwidth}\PY{p}{,} \PY{n}{kernel}\PY{o}{=}\PY{l+s+s1}{\PYZsq{}}\PY{l+s+s1}{gaussian}\PY{l+s+s1}{\PYZsq{}}\PY{p}{)}
\PY{n}{kde}\PY{o}{.}\PY{n}{fit}\PY{p}{(}\PY{n}{samples}\PY{p}{)}

\PY{c+c1}{\PYZsh{} score\PYZus{}samples returns the log of the probability density}
\PY{n}{x\PYZus{}d} \PY{o}{=} \PY{n}{np}\PY{o}{.}\PY{n}{linspace}\PY{p}{(}\PY{n}{x\PYZus{}min}\PY{p}{,} \PY{n}{x\PYZus{}max}\PY{p}{,} \PY{n}{n\PYZus{}samples}\PY{p}{)}\PY{o}{.}\PY{n}{reshape}\PY{p}{(}\PY{o}{\PYZhy{}}\PY{l+m+mi}{1}\PY{p}{,} \PY{l+m+mi}{1}\PY{p}{)}
\PY{n}{logprob} \PY{o}{=} \PY{n}{kde}\PY{o}{.}\PY{n}{score\PYZus{}samples}\PY{p}{(}\PY{n}{x\PYZus{}d}\PY{p}{)}
\end{Verbatim}
\end{tcolorbox}

    \begin{center}
    \adjustimage{max size={0.9\linewidth}{0.9\paperheight}}{KDE_files/KDE_18_0.png}
    \end{center}
    { \hspace*{\fill} \\}
    
    The bell shape is now smoother, though we're loosing detail on the
second ``bump'' over the value of ``8''. Probably we should re-check
what is the optimal bandwidth again with this number of samples.

    \begin{tcolorbox}[breakable, size=fbox, boxrule=1pt, pad at break*=1mm,colback=cellbackground, colframe=cellborder]
\prompt{In}{incolor}{12}{\boxspacing}
\begin{Verbatim}[commandchars=\\\{\}]
\PY{n}{bandwidths} \PY{o}{=} \PY{l+m+mi}{10} \PY{o}{*}\PY{o}{*} \PY{n}{np}\PY{o}{.}\PY{n}{linspace}\PY{p}{(}\PY{o}{\PYZhy{}}\PY{l+m+mi}{1}\PY{p}{,} \PY{l+m+mi}{1}\PY{p}{,} \PY{l+m+mi}{100}\PY{p}{)}
\PY{n}{grid} \PY{o}{=} \PY{n}{GridSearchCV}\PY{p}{(}\PY{n}{KernelDensity}\PY{p}{(}\PY{n}{kernel}\PY{o}{=}\PY{l+s+s1}{\PYZsq{}}\PY{l+s+s1}{gaussian}\PY{l+s+s1}{\PYZsq{}}\PY{p}{)}\PY{p}{,}
                    \PY{p}{\PYZob{}}\PY{l+s+s1}{\PYZsq{}}\PY{l+s+s1}{bandwidth}\PY{l+s+s1}{\PYZsq{}}\PY{p}{:} \PY{n}{bandwidths}\PY{p}{\PYZcb{}}\PY{p}{,}
                    \PY{n}{cv}\PY{o}{=}\PY{n}{LeaveOneOut}\PY{p}{(}\PY{p}{)}\PY{p}{)}
\PY{n}{grid}\PY{o}{.}\PY{n}{fit}\PY{p}{(}\PY{n}{samples}\PY{p}{)}\PY{p}{;}
\PY{n}{bandwidth} \PY{o}{=} \PY{n}{grid}\PY{o}{.}\PY{n}{best\PYZus{}params\PYZus{}}\PY{p}{[}\PY{l+s+s1}{\PYZsq{}}\PY{l+s+s1}{bandwidth}\PY{l+s+s1}{\PYZsq{}}\PY{p}{]}
\PY{n+nb}{print}\PY{p}{(}\PY{l+s+s1}{\PYZsq{}}\PY{l+s+s1}{Optimal Bandwidth: }\PY{l+s+si}{\PYZob{}:.2f\PYZcb{}}\PY{l+s+s1}{\PYZsq{}}\PY{o}{.}\PY{n}{format}\PY{p}{(}\PY{n}{bandwidth}\PY{p}{)}\PY{p}{)}
\end{Verbatim}
\end{tcolorbox}

    \begin{Verbatim}[commandchars=\\\{\}]
Optimal Bandwidth: 0.71
    \end{Verbatim}

    Plot the results with the optimal bandwidth obtained for the 200
samples.

    \begin{tcolorbox}[breakable, size=fbox, boxrule=1pt, pad at break*=1mm,colback=cellbackground, colframe=cellborder]
\prompt{In}{incolor}{13}{\boxspacing}
\begin{Verbatim}[commandchars=\\\{\}]
\PY{c+c1}{\PYZsh{} instantiate and fit the KDE model}
\PY{n}{kde} \PY{o}{=} \PY{n}{KernelDensity}\PY{p}{(}\PY{n}{bandwidth}\PY{o}{=}\PY{n}{bandwidth}\PY{p}{,} \PY{n}{kernel}\PY{o}{=}\PY{l+s+s1}{\PYZsq{}}\PY{l+s+s1}{gaussian}\PY{l+s+s1}{\PYZsq{}}\PY{p}{)}
\PY{n}{kde}\PY{o}{.}\PY{n}{fit}\PY{p}{(}\PY{n}{samples}\PY{p}{)}

\PY{c+c1}{\PYZsh{} score\PYZus{}samples returns the log of the probability density}
\PY{n}{x\PYZus{}d} \PY{o}{=} \PY{n}{np}\PY{o}{.}\PY{n}{linspace}\PY{p}{(}\PY{n}{x\PYZus{}min}\PY{p}{,} \PY{n}{x\PYZus{}max}\PY{p}{,} \PY{n}{n\PYZus{}samples}\PY{p}{)}\PY{o}{.}\PY{n}{reshape}\PY{p}{(}\PY{o}{\PYZhy{}}\PY{l+m+mi}{1}\PY{p}{,} \PY{l+m+mi}{1}\PY{p}{)}
\PY{n}{logprob} \PY{o}{=} \PY{n}{kde}\PY{o}{.}\PY{n}{score\PYZus{}samples}\PY{p}{(}\PY{n}{x\PYZus{}d}\PY{p}{)}
\end{Verbatim}
\end{tcolorbox}

    \begin{center}
    \adjustimage{max size={0.9\linewidth}{0.9\paperheight}}{KDE_files/KDE_23_0.png}
    \end{center}
    { \hspace*{\fill} \\}
    
    Now we can see that the optimal bandwidth over the specific 200 samples
used builds an almost perfect fit over the distribution of the data.

    \hypertarget{kde-classifier-over-hr-dataset}{%
\subsection{KDE Classifier over HR
dataset}\label{kde-classifier-over-hr-dataset}}

Mostly from
\href{https://github.com/jakevdp/PythonDataScienceHandbook/blob/master/notebooks/05.13-Kernel-Density-Estimation.ipynb}{here}.
Let's use a KDE Classifier to solve the classification problem in the HR
dataset.

The general approach for \textbf{generative classification} is this:

\begin{enumerate}
\def\labelenumi{\arabic{enumi}.}
\tightlist
\item
  Split the training data by label.
\item
  For each set, fit a KDE to obtain a generative model of the data. This
  allows you for any observation \(x\) and label \(y\) to compute a
  likelihood \(P(x~|~y)\).
\item
  From the number of examples of each class in the training set, compute
  the class prior, \(P(y)\).
\item
  For an unknown point \(x\), the posterior probability for each class
  is \(P(y~|~x) \propto P(x~|~y)P(y)\). The class which maximizes this
  posterior is the label assigned to the point.
\end{enumerate}

The algorithm is straightforward and intuitive to understand; the more
difficult piece is couching it within the Scikit-Learn framework in
order to make use of the grid search and cross-validation architecture.

    \hypertarget{load-the-hr-dataset}{%
\subsubsection{Load the HR dataset}\label{load-the-hr-dataset}}

Perform basic transformation over the features. We will only use
numerical features.

    \begin{tcolorbox}[breakable, size=fbox, boxrule=1pt, pad at break*=1mm,colback=cellbackground, colframe=cellborder]
\prompt{In}{incolor}{15}{\boxspacing}
\begin{Verbatim}[commandchars=\\\{\}]
\PY{n}{hr} \PY{o}{=} \PY{n}{Dataset}\PY{p}{(}\PY{l+s+s1}{\PYZsq{}}\PY{l+s+s1}{./data/hr\PYZhy{}analytics.zip}\PY{l+s+s1}{\PYZsq{}}\PY{p}{)}
\PY{n}{hr}\PY{o}{.}\PY{n}{set\PYZus{}target}\PY{p}{(}\PY{l+s+s1}{\PYZsq{}}\PY{l+s+s1}{left}\PY{l+s+s1}{\PYZsq{}}\PY{p}{)}
\PY{n}{hr}\PY{o}{.}\PY{n}{to\PYZus{}categorical}\PY{p}{(}\PY{p}{[}\PY{l+s+s1}{\PYZsq{}}\PY{l+s+s1}{number\PYZus{}project}\PY{l+s+s1}{\PYZsq{}}\PY{p}{,} \PY{l+s+s1}{\PYZsq{}}\PY{l+s+s1}{time\PYZus{}spend\PYZus{}company}\PY{l+s+s1}{\PYZsq{}}\PY{p}{,}
                   \PY{l+s+s1}{\PYZsq{}}\PY{l+s+s1}{promotion\PYZus{}last\PYZus{}5years}\PY{l+s+s1}{\PYZsq{}}\PY{p}{,} \PY{l+s+s1}{\PYZsq{}}\PY{l+s+s1}{Work\PYZus{}accident}\PY{l+s+s1}{\PYZsq{}}\PY{p}{]}\PY{p}{)}
\PY{n}{hr}\PY{o}{.}\PY{n}{scale}\PY{p}{(}\PY{p}{)}
\PY{n}{hr}\PY{o}{.}\PY{n}{summary}\PY{p}{(}\PY{l+s+s1}{\PYZsq{}}\PY{l+s+s1}{numerical}\PY{l+s+s1}{\PYZsq{}}\PY{p}{)}
\end{Verbatim}
\end{tcolorbox}

    \begin{Verbatim}[commandchars=\\\{\}]
Features Summary (numerical):
'satisfaction\_level'  : float64    Min.(-2.1) 1stQ(-0.6) Med.(0.10) Mean(2.88)
3rdQ(0.83) Max.(1.55)
'last\_evaluation'     : float64    Min.(-2.0) 1stQ(-0.9) Med.(0.02) Mean(-3.9)
3rdQ(0.89) Max.(1.65)
'average\_montly\_hours': float64    Min.(-2.1) 1stQ(-0.9) Med.(-0.0) Mean(-8.7)
3rdQ(0.88) Max.(2.18)
    \end{Verbatim}

    \begin{tcolorbox}[breakable, size=fbox, boxrule=1pt, pad at break*=1mm,colback=cellbackground, colframe=cellborder]
\prompt{In}{incolor}{16}{\boxspacing}
\begin{Verbatim}[commandchars=\\\{\}]
\PY{k}{class} \PY{n+nc}{KDEClassifier}\PY{p}{(}\PY{n}{BaseEstimator}\PY{p}{,} \PY{n}{ClassifierMixin}\PY{p}{)}\PY{p}{:}
    \PY{l+s+sd}{\PYZdq{}\PYZdq{}\PYZdq{}Bayesian generative classification based on KDE}
\PY{l+s+sd}{    }
\PY{l+s+sd}{    Parameters}
\PY{l+s+sd}{    \PYZhy{}\PYZhy{}\PYZhy{}\PYZhy{}\PYZhy{}\PYZhy{}\PYZhy{}\PYZhy{}\PYZhy{}\PYZhy{}}
\PY{l+s+sd}{    bandwidth : float}
\PY{l+s+sd}{        the kernel bandwidth within each class}
\PY{l+s+sd}{    kernel : str}
\PY{l+s+sd}{        the kernel name, passed to KernelDensity}
\PY{l+s+sd}{    \PYZdq{}\PYZdq{}\PYZdq{}}
    \PY{k}{def} \PY{n+nf+fm}{\PYZus{}\PYZus{}init\PYZus{}\PYZus{}}\PY{p}{(}\PY{n+nb+bp}{self}\PY{p}{,} \PY{n}{bandwidth}\PY{o}{=}\PY{l+m+mf}{1.0}\PY{p}{,} \PY{n}{kernel}\PY{o}{=}\PY{l+s+s1}{\PYZsq{}}\PY{l+s+s1}{gaussian}\PY{l+s+s1}{\PYZsq{}}\PY{p}{)}\PY{p}{:}
        \PY{n+nb+bp}{self}\PY{o}{.}\PY{n}{bandwidth} \PY{o}{=} \PY{n}{bandwidth}
        \PY{n+nb+bp}{self}\PY{o}{.}\PY{n}{kernel} \PY{o}{=} \PY{n}{kernel}
        
    \PY{k}{def} \PY{n+nf}{fit}\PY{p}{(}\PY{n+nb+bp}{self}\PY{p}{,} \PY{n}{X}\PY{p}{,} \PY{n}{y}\PY{p}{)}\PY{p}{:}
        \PY{k}{if} \PY{n+nb}{isinstance}\PY{p}{(}\PY{n}{y}\PY{p}{,} \PY{n}{pd}\PY{o}{.}\PY{n}{DataFrame}\PY{p}{)}\PY{p}{:}
            \PY{n}{y} \PY{o}{=} \PY{n}{y}\PY{o}{.}\PY{n}{values}
        \PY{n+nb+bp}{self}\PY{o}{.}\PY{n}{classes\PYZus{}} \PY{o}{=} \PY{n}{np}\PY{o}{.}\PY{n}{sort}\PY{p}{(}\PY{n}{np}\PY{o}{.}\PY{n}{unique}\PY{p}{(}\PY{n}{y}\PY{p}{)}\PY{p}{)}
        \PY{n}{training\PYZus{}sets} \PY{o}{=} \PY{p}{[}\PY{n}{X}\PY{p}{[}\PY{n}{y} \PY{o}{==} \PY{n}{yi}\PY{p}{]} \PY{k}{for} \PY{n}{yi} \PY{o+ow}{in} \PY{n+nb+bp}{self}\PY{o}{.}\PY{n}{classes\PYZus{}}\PY{p}{]}
        \PY{n+nb+bp}{self}\PY{o}{.}\PY{n}{models\PYZus{}} \PY{o}{=} \PY{p}{[}\PY{n}{KernelDensity}\PY{p}{(}\PY{n}{bandwidth}\PY{o}{=}\PY{n+nb+bp}{self}\PY{o}{.}\PY{n}{bandwidth}\PY{p}{,}
                                      \PY{n}{kernel}\PY{o}{=}\PY{n+nb+bp}{self}\PY{o}{.}\PY{n}{kernel}\PY{p}{)}\PY{o}{.}\PY{n}{fit}\PY{p}{(}\PY{n}{Xi}\PY{p}{)}
                        \PY{k}{for} \PY{n}{Xi} \PY{o+ow}{in} \PY{n}{training\PYZus{}sets}\PY{p}{]}
        \PY{n+nb+bp}{self}\PY{o}{.}\PY{n}{logpriors\PYZus{}} \PY{o}{=} \PY{p}{[}\PY{n}{np}\PY{o}{.}\PY{n}{log}\PY{p}{(}\PY{n}{Xi}\PY{o}{.}\PY{n}{shape}\PY{p}{[}\PY{l+m+mi}{0}\PY{p}{]} \PY{o}{/} \PY{n}{X}\PY{o}{.}\PY{n}{shape}\PY{p}{[}\PY{l+m+mi}{0}\PY{p}{]}\PY{p}{)}
                           \PY{k}{for} \PY{n}{Xi} \PY{o+ow}{in} \PY{n}{training\PYZus{}sets}\PY{p}{]}
        \PY{k}{return} \PY{n+nb+bp}{self}
        
    \PY{k}{def} \PY{n+nf}{predict\PYZus{}proba}\PY{p}{(}\PY{n+nb+bp}{self}\PY{p}{,} \PY{n}{X}\PY{p}{)}\PY{p}{:}
        \PY{n}{logprobs} \PY{o}{=} \PY{n}{np}\PY{o}{.}\PY{n}{array}\PY{p}{(}\PY{p}{[}\PY{n}{model}\PY{o}{.}\PY{n}{score\PYZus{}samples}\PY{p}{(}\PY{n}{X}\PY{p}{)}
                             \PY{k}{for} \PY{n}{model} \PY{o+ow}{in} \PY{n+nb+bp}{self}\PY{o}{.}\PY{n}{models\PYZus{}}\PY{p}{]}\PY{p}{)}\PY{o}{.}\PY{n}{T}
        \PY{n}{result} \PY{o}{=} \PY{n}{np}\PY{o}{.}\PY{n}{exp}\PY{p}{(}\PY{n}{logprobs} \PY{o}{+} \PY{n+nb+bp}{self}\PY{o}{.}\PY{n}{logpriors\PYZus{}}\PY{p}{)}
        \PY{k}{return} \PY{n}{result} \PY{o}{/} \PY{n}{result}\PY{o}{.}\PY{n}{sum}\PY{p}{(}\PY{l+m+mi}{1}\PY{p}{,} \PY{n}{keepdims}\PY{o}{=}\PY{k+kc}{True}\PY{p}{)}
        
    \PY{k}{def} \PY{n+nf}{predict}\PY{p}{(}\PY{n+nb+bp}{self}\PY{p}{,} \PY{n}{X}\PY{p}{)}\PY{p}{:}
        \PY{k}{return} \PY{n+nb+bp}{self}\PY{o}{.}\PY{n}{classes\PYZus{}}\PY{p}{[}\PY{n}{np}\PY{o}{.}\PY{n}{argmax}\PY{p}{(}\PY{n+nb+bp}{self}\PY{o}{.}\PY{n}{predict\PYZus{}proba}\PY{p}{(}\PY{n}{X}\PY{p}{)}\PY{p}{,} \PY{l+m+mi}{1}\PY{p}{)}\PY{p}{]}
\end{Verbatim}
\end{tcolorbox}

    \begin{tcolorbox}[breakable, size=fbox, boxrule=1pt, pad at break*=1mm,colback=cellbackground, colframe=cellborder]
\prompt{In}{incolor}{17}{\boxspacing}
\begin{Verbatim}[commandchars=\\\{\}]
\PY{n}{bandwidths} \PY{o}{=} \PY{l+m+mi}{10} \PY{o}{*}\PY{o}{*} \PY{n}{np}\PY{o}{.}\PY{n}{linspace}\PY{p}{(}\PY{o}{\PYZhy{}}\PY{l+m+mi}{1}\PY{p}{,} \PY{l+m+mi}{0}\PY{p}{,} \PY{l+m+mi}{10}\PY{p}{)}
\PY{n}{grid} \PY{o}{=} \PY{n}{GridSearchCV}\PY{p}{(}\PY{n}{KDEClassifier}\PY{p}{(}\PY{p}{)}\PY{p}{,} \PY{p}{\PYZob{}}\PY{l+s+s1}{\PYZsq{}}\PY{l+s+s1}{bandwidth}\PY{l+s+s1}{\PYZsq{}}\PY{p}{:} \PY{n}{bandwidths}\PY{p}{\PYZcb{}}\PY{p}{,}
                   \PY{n}{scoring}\PY{o}{=}\PY{l+s+s1}{\PYZsq{}}\PY{l+s+s1}{accuracy}\PY{l+s+s1}{\PYZsq{}}\PY{p}{)}
\PY{n}{grid}\PY{o}{.}\PY{n}{fit}\PY{p}{(}\PY{n}{hr}\PY{o}{.}\PY{n}{select}\PY{p}{(}\PY{l+s+s1}{\PYZsq{}}\PY{l+s+s1}{numerical}\PY{l+s+s1}{\PYZsq{}}\PY{p}{)}\PY{p}{,} \PY{n}{hr}\PY{o}{.}\PY{n}{target}\PY{p}{)}
\PY{n}{scores} \PY{o}{=} \PY{n}{grid}\PY{o}{.}\PY{n}{cv\PYZus{}results\PYZus{}}\PY{p}{[}\PY{l+s+s1}{\PYZsq{}}\PY{l+s+s1}{mean\PYZus{}test\PYZus{}score}\PY{l+s+s1}{\PYZsq{}}\PY{p}{]}
\end{Verbatim}
\end{tcolorbox}

    \begin{center}
    \adjustimage{max size={0.9\linewidth}{0.9\paperheight}}{KDE_files/KDE_30_0.png}
    \end{center}
    { \hspace*{\fill} \\}
    
    We see that this not-so-naive Bayesian KDE classifier reaches a
cross-validation accuracy of just over 93\%; this is compared to around
76\% for the naive Bayesian classification:

    \begin{tcolorbox}[breakable, size=fbox, boxrule=1pt, pad at break*=1mm,colback=cellbackground, colframe=cellborder]
\prompt{In}{incolor}{19}{\boxspacing}
\begin{Verbatim}[commandchars=\\\{\}]
\PY{n}{cross\PYZus{}val\PYZus{}score}\PY{p}{(}\PY{n}{GaussianNB}\PY{p}{(}\PY{p}{)}\PY{p}{,} \PY{n}{hr}\PY{o}{.}\PY{n}{select}\PY{p}{(}\PY{l+s+s1}{\PYZsq{}}\PY{l+s+s1}{numerical}\PY{l+s+s1}{\PYZsq{}}\PY{p}{)}\PY{p}{,} \PY{n}{hr}\PY{o}{.}\PY{n}{target}\PY{p}{)}\PY{o}{.}\PY{n}{mean}\PY{p}{(}\PY{p}{)}
\end{Verbatim}
\end{tcolorbox}

            \begin{tcolorbox}[breakable, size=fbox, boxrule=.5pt, pad at break*=1mm, opacityfill=0]
\prompt{Out}{outcolor}{19}{\boxspacing}
\begin{Verbatim}[commandchars=\\\{\}]
0.763850639101923
\end{Verbatim}
\end{tcolorbox}
        
    \hypertarget{hr-example}{%
\subsection{HR Example}\label{hr-example}}

Load and perform the basic transofrmations. In our case, we will try to
solve the problem using only the true numerical features in it.

    \begin{tcolorbox}[breakable, size=fbox, boxrule=1pt, pad at break*=1mm,colback=cellbackground, colframe=cellborder]
\prompt{In}{incolor}{20}{\boxspacing}
\begin{Verbatim}[commandchars=\\\{\}]
\PY{n}{hr} \PY{o}{=} \PY{n}{Dataset}\PY{p}{(}\PY{l+s+s1}{\PYZsq{}}\PY{l+s+s1}{data/hr\PYZhy{}analytics.zip}\PY{l+s+s1}{\PYZsq{}}\PY{p}{)}
\PY{n}{hr}\PY{o}{.}\PY{n}{set\PYZus{}target}\PY{p}{(}\PY{l+s+s1}{\PYZsq{}}\PY{l+s+s1}{left}\PY{l+s+s1}{\PYZsq{}}\PY{p}{)}
\PY{n}{hr}\PY{o}{.}\PY{n}{to\PYZus{}categorical}\PY{p}{(}\PY{p}{[}\PY{l+s+s1}{\PYZsq{}}\PY{l+s+s1}{number\PYZus{}project}\PY{l+s+s1}{\PYZsq{}}\PY{p}{,} \PY{l+s+s1}{\PYZsq{}}\PY{l+s+s1}{time\PYZus{}spend\PYZus{}company}\PY{l+s+s1}{\PYZsq{}}\PY{p}{,}
                   \PY{l+s+s1}{\PYZsq{}}\PY{l+s+s1}{promotion\PYZus{}last\PYZus{}5years}\PY{l+s+s1}{\PYZsq{}}\PY{p}{,} \PY{l+s+s1}{\PYZsq{}}\PY{l+s+s1}{Work\PYZus{}accident}\PY{l+s+s1}{\PYZsq{}}\PY{p}{]}\PY{p}{)}
\PY{n}{hr}\PY{o}{.}\PY{n}{drop\PYZus{}columns}\PY{p}{(}\PY{n}{hr}\PY{o}{.}\PY{n}{names}\PY{p}{(}\PY{l+s+s1}{\PYZsq{}}\PY{l+s+s1}{categorical}\PY{l+s+s1}{\PYZsq{}}\PY{p}{)}\PY{p}{)}
\PY{n}{hr}\PY{o}{.}\PY{n}{summary}\PY{p}{(}\PY{p}{)}
\end{Verbatim}
\end{tcolorbox}

    \begin{Verbatim}[commandchars=\\\{\}]
Features Summary (all):
'satisfaction\_level'  : float64    Min.(0.09) 1stQ(0.44) Med.(0.64) Mean(0.61)
3rdQ(0.82) Max.(1.0)
'last\_evaluation'     : float64    Min.(0.36) 1stQ(0.56) Med.(0.72) Mean(0.71)
3rdQ(0.87) Max.(1.0)
'average\_montly\_hours': float64    Min.(96.0) 1stQ(156.) Med.(200.) Mean(201.)
3rdQ(245.) Max.(310.)
'left'                : float64    Min.(0.0) 1stQ(0.0) Med.(0.0) Mean(0.23)
3rdQ(0.0) Max.(1.0)
    \end{Verbatim}

\hypertarget{train-the-classifier-and-estimate-the-performance}{%
\subsubsection{Train the classifier and estimate the
performance}\label{train-the-classifier-and-estimate-the-performance}}

    \begin{tcolorbox}[breakable, size=fbox, boxrule=1pt, pad at break*=1mm,colback=cellbackground, colframe=cellborder]
\prompt{In}{incolor}{21}{\boxspacing}
\begin{Verbatim}[commandchars=\\\{\}]
\PY{n}{X}\PY{p}{,} \PY{n}{y} \PY{o}{=} \PY{n}{hr}\PY{o}{.}\PY{n}{split}\PY{p}{(}\PY{p}{)}

\PY{n}{model} \PY{o}{=} \PY{n}{KDEClassifier}\PY{p}{(}\PY{n}{bandwidth}\PY{o}{=}\PY{l+m+mf}{0.1}\PY{p}{)}
\PY{n}{model}\PY{o}{.}\PY{n}{fit}\PY{p}{(}\PY{n}{X}\PY{o}{.}\PY{n}{train}\PY{p}{,} \PY{n}{y}\PY{o}{.}\PY{n}{train}\PY{p}{)}
\PY{n}{cv\PYZus{}accuracies} \PY{o}{=} \PY{n}{cross\PYZus{}val\PYZus{}score}\PY{p}{(}\PY{n}{model}\PY{p}{,} \PY{n}{X}\PY{o}{.}\PY{n}{train}\PY{p}{,} \PY{n}{y}\PY{o}{.}\PY{n}{train}\PY{p}{,} \PY{n}{cv}\PY{o}{=}\PY{l+m+mi}{5}\PY{p}{,} \PY{n}{scoring}\PY{o}{=}\PY{l+s+s1}{\PYZsq{}}\PY{l+s+s1}{accuracy}\PY{l+s+s1}{\PYZsq{}}\PY{p}{)}
\PY{n+nb}{print}\PY{p}{(}\PY{l+s+s1}{\PYZsq{}}\PY{l+s+s1}{Avg. Acc: }\PY{l+s+si}{\PYZob{}:.4f\PYZcb{}}\PY{l+s+s1}{ +/\PYZhy{} }\PY{l+s+si}{\PYZob{}:.4f\PYZcb{}}\PY{l+s+s1}{\PYZsq{}}\PY{o}{.}\PY{n}{format}\PY{p}{(}
    \PY{n}{np}\PY{o}{.}\PY{n}{mean}\PY{p}{(}\PY{n}{cv\PYZus{}accuracies}\PY{p}{)}\PY{p}{,} \PY{n}{np}\PY{o}{.}\PY{n}{std}\PY{p}{(}\PY{n}{cv\PYZus{}accuracies}\PY{p}{)}\PY{p}{)}\PY{p}{)}
\end{Verbatim}
\end{tcolorbox}

    \begin{Verbatim}[commandchars=\\\{\}]
Avg. Acc: 0.9019 +/- 0.0041
    \end{Verbatim}

    Obtain different metrics

    \begin{tcolorbox}[breakable, size=fbox, boxrule=1pt, pad at break*=1mm,colback=cellbackground, colframe=cellborder]
\prompt{In}{incolor}{22}{\boxspacing}
\begin{Verbatim}[commandchars=\\\{\}]
\PY{n}{y\PYZus{}pred} \PY{o}{=} \PY{n}{model}\PY{o}{.}\PY{n}{predict}\PY{p}{(}\PY{n}{X}\PY{o}{.}\PY{n}{test}\PY{p}{)}
\PY{n}{auc} \PY{o}{=} \PY{n}{roc\PYZus{}auc\PYZus{}score}\PY{p}{(}\PY{n}{y}\PY{o}{.}\PY{n}{test}\PY{p}{,} \PY{n}{y\PYZus{}pred}\PY{p}{)}
\PY{n}{acc} \PY{o}{=} \PY{n}{accuracy\PYZus{}score}\PY{p}{(}\PY{n}{y}\PY{o}{.}\PY{n}{test}\PY{p}{,} \PY{n}{y\PYZus{}pred}\PY{p}{)}

\PY{n+nb}{print}\PY{p}{(}\PY{l+s+s1}{\PYZsq{}}\PY{l+s+s1}{AuC: }\PY{l+s+si}{\PYZob{}:.4f\PYZcb{}}\PY{l+s+s1}{\PYZsq{}}\PY{o}{.}\PY{n}{format}\PY{p}{(}\PY{n}{auc}\PY{p}{)}\PY{p}{)}
\PY{n+nb}{print}\PY{p}{(}\PY{l+s+s1}{\PYZsq{}}\PY{l+s+s1}{Acc: }\PY{l+s+si}{\PYZob{}:.4f\PYZcb{}}\PY{l+s+s1}{\PYZsq{}}\PY{o}{.}\PY{n}{format}\PY{p}{(}\PY{n}{acc}\PY{p}{)}\PY{p}{)}
\end{Verbatim}
\end{tcolorbox}

    \begin{Verbatim}[commandchars=\\\{\}]
AuC: 0.8707
Acc: 0.9123
    \end{Verbatim}

	\hypertarget{ROC-curve}{%
	\subsubsection{ROC Curve}\label{ROC-curve}}

Let's simply plot the ROC curve with the probabilities obtained by our model, and retrieved via \texttt{predict\_proba()}


    \begin{tcolorbox}[breakable, size=fbox, boxrule=1pt, pad at break*=1mm,colback=cellbackground, colframe=cellborder]
\prompt{In}{incolor}{23}{\boxspacing}
\begin{Verbatim}[commandchars=\\\{\}]
\PY{c+c1}{\PYZsh{} roc curve and auc}
\PY{n}{probs} \PY{o}{=} \PY{n}{model}\PY{o}{.}\PY{n}{predict\PYZus{}proba}\PY{p}{(}\PY{n}{X}\PY{o}{.}\PY{n}{test}\PY{p}{)}
\PY{c+c1}{\PYZsh{} keep probabilities for the positive outcome only}
\PY{n}{probs} \PY{o}{=} \PY{n}{probs}\PY{p}{[}\PY{p}{:}\PY{p}{,} \PY{l+m+mi}{1}\PY{p}{]}
\PY{c+c1}{\PYZsh{} calculate AUC}
\PY{n}{auc} \PY{o}{=} \PY{n}{roc\PYZus{}auc\PYZus{}score}\PY{p}{(}\PY{n}{y}\PY{o}{.}\PY{n}{test}\PY{p}{,} \PY{n}{probs}\PY{p}{)}
\PY{n+nb}{print}\PY{p}{(}\PY{l+s+s1}{\PYZsq{}}\PY{l+s+s1}{AUC: }\PY{l+s+si}{\PYZpc{}.3f}\PY{l+s+s1}{\PYZsq{}} \PY{o}{\PYZpc{}} \PY{n}{auc}\PY{p}{)}

\PY{c+c1}{\PYZsh{} calculate roc curve}
\PY{n}{fpr}\PY{p}{,} \PY{n}{tpr}\PY{p}{,} \PY{n}{thresholds} \PY{o}{=} \PY{n}{roc\PYZus{}curve}\PY{p}{(}\PY{n}{y}\PY{o}{.}\PY{n}{test}\PY{p}{,} \PY{n}{probs}\PY{p}{)}
\PY{n}{plt}\PY{o}{.}\PY{n}{figure}\PY{p}{(}\PY{n}{figsize}\PY{o}{=}\PY{p}{(}\PY{l+m+mi}{10}\PY{p}{,} \PY{l+m+mi}{5}\PY{p}{)}\PY{p}{)}
\PY{n}{plt}\PY{o}{.}\PY{n}{plot}\PY{p}{(}\PY{p}{[}\PY{l+m+mi}{0}\PY{p}{,} \PY{l+m+mi}{1}\PY{p}{]}\PY{p}{,} \PY{p}{[}\PY{l+m+mi}{0}\PY{p}{,} \PY{l+m+mi}{1}\PY{p}{]}\PY{p}{,} \PY{n}{linestyle}\PY{o}{=}\PY{l+s+s1}{\PYZsq{}}\PY{l+s+s1}{\PYZhy{}\PYZhy{}}\PY{l+s+s1}{\PYZsq{}}\PY{p}{)}
\PY{n}{plt}\PY{o}{.}\PY{n}{plot}\PY{p}{(}\PY{n}{fpr}\PY{p}{,} \PY{n}{tpr}\PY{p}{,} \PY{n}{marker}\PY{o}{=}\PY{l+s+s1}{\PYZsq{}}\PY{l+s+s1}{.}\PY{l+s+s1}{\PYZsq{}}\PY{p}{)}
\PY{n}{plt}\PY{o}{.}\PY{n}{show}\PY{p}{(}\PY{p}{)}\PY{p}{;}
\end{Verbatim}
\end{tcolorbox}

    \begin{Verbatim}[commandchars=\\\{\}]
AUC: 0.959
    \end{Verbatim}

    \begin{center}
    \adjustimage{max size={0.9\linewidth}{0.9\paperheight}}{KDE_files/KDE_40_1.png}
    \end{center}
    { \hspace*{\fill} \\}
    

    % Add a bibliography block to the postdoc
    
    
    
\end{document}
